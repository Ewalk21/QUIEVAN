\documentclass[aps,pra,notitlepage,amsmath,amssymb,letterpaper,12pt]{revtex4-1}
\usepackage{amsthm,enumerate,amsmath}
\usepackage{graphicx}
%  Above uses the Americal Physical Society template for Physical Review A
%  as a reasonable and fully-featured default template

%  Below define helpful commands to set up problem environments easily
\newtheorem{theorem}{Theorem}
\newenvironment{problem}[2][Problem]{\begin{trivlist}
\item[\hskip \labelsep {\bfseries #1}\hskip \labelsep {\bfseries #2.}]}{\end{trivlist}}
\newenvironment{solution}{\begin{proof}[Solution]}{\end{proof}}

% --------------------------------------------------------------
%                   Document Begins Here
% --------------------------------------------------------------

\begin{document}

\title{Final Proofs for MATH450 Real Analysis}
\author{Quinn Gates, Evan Walker}
%\affiliation{Math 450, Schmid College of Science and Technology, Chapman University}
\date{\today}

\maketitle


\begin{theorem}[3.24 Bolzano-Weierstrass]
If a bounded set $S \subseteq \mathbb{R}^n$ contains infinitely many points, then there is at least one point in $\mathbb{R}^n$ which is an accumulation point of $S$.
%Application: Thm. 3.25 Cantor Intersection Theorem
\end{theorem}

\begin{proof}
\end{proof}


\begin{theorem}[3.27] \nonumber
Let $G = \{A_{1},A_{2},...\}$ denote the countable collection of all n-balls having rational radii and centers at points with rational coordinates. Assume $x \in \mathbb{R}^n$ and let $S$ be an open set in $\mathbb{R}^n$ and $x \in S$. Then $x \in A_{k} \subseteq S$ for some k. That is, x is contained by some n-ball in $G$ which is contained by $S$.
%Application: Thm 3.28 (Lindeloff Covering Theorem)
\end{theorem}

\begin{proof}
\end{proof}


\begin{theorem}[3.29 Heine-Borel]
Let $F$ be an open covering of a closed and bounded set $A \subseteq \mathbb{R}^n$. Then a finite subcollection of $F$ also covers $A$.
%Application: you choose
\end{theorem}

\begin{proof}
\end{proof}


\begin{theorem}[3.31]
Let $S \subseteq \mathbb{R}^n$. Then the following statements are equivalent
\begin{enumerate}[\upshape a)]
  \item $S$ is compact. %note a implies b by heineborel
  \item $S$ is closed and bounded.
  \item Every infinite subset of $S$ has an accumulation point in $S$.
\end{enumerate}
%comparison with metric spaces (without proof)
\end{theorem}

\begin{proof}
\end{proof}

\newpage{}
\begin{theorem}
Continuous Functions in Metric Spaces: Thm. 4.16, Thm. 4.23, Thm. 4.25
\end{theorem}

\begin{proof}
\end{proof}


\begin{theorem}[Uniform Continuity Thm. 4.47 (Heine)]
Let $f:S \to T$ be a function from one metric space $(S,d_{S})$ to another $(T,d_{T})$. Let $A$ be a compact subset of $S$ and assume that $f$ is continuous on $A$. Then $f$ is uniformly continuous on $A$.
\end{theorem}

\begin{proof}
\end{proof}


\begin{theorem}[Thm 5.10 (Rolle) and Thm 5.12 (Generalized Mean Value)]
Let $f$ and $g$ be two functions, each having a derivative (finite or infinite) at each point
of an open interval $(a,b)$ and each continuous at the endpoints $a$ and $b$. if $f(a)=f(b)$, then there is at least one interior point c at which
\[f^\prime(c) = 0\]
Now assume also that there is no interior point $x$ at which both $f^\prime(x)$ and $g^\prime(x)$ are infinite. Then for some interior point $c$ we have
\[f^\prime(c)[g(b)-g(a)] = g^\prime(c)[f(b)-f(a)]\]
%Application: Thm 5.11
\end{theorem}

\begin{proof}
%proof for rolles

%proof for 5.12
Let $h(x) = f(x)[g(b) - g(a)] - g(x)[f(b) - f(a)]$. Since $h(x)$ is a linear combination of $f(x)$ and $g(x)$, $h(x)$ inherits many of the assumptions about $f(x)$ and $g(x)$, which is exactly what we want in order to use Rolle's theorem. Note that,
\begin{align*}
$h(a)$ &= $ f(a)[g(b) - g(a)] - g(a)[f(b) - f(a)]$
\\ &= $ f(a)g(b) - g(a)f(b)$
\\ &= $ - f(b)g(a) + g(b)f(a)$
\\ &= $ f(b)[g(b) - g(a)] - g(b)[f(b) - f(a)]$
\\ &= $ h(b)$
\end{align*}
So $h(a) = h(b)$, $ \Rightarrow $ by Rolle's theorem we have that $\exists $ $c$ such that $ h^\prime(c) = 0 $. And Hence,
\[h^\prime(c) = f^\prime(c)[g(b) - g(a)] - g^\prime(c)[f(b) - f(a)] = 0\]
$ \Rightarrow $
\[f^\prime(c)[g(b)-g(a)] = g^\prime(c)[f(b)-f(a)]\]
This proves the assertion.
\end{proof}

%\section{The following three proofs are with respect to problem 8}
The following three proofs are with respect to problem 8
\begin{theorem}[7.8]
Assume $f \in R(\alpha)$ on $[a,b]$ and assume that $\alpha$ has a continuous derivative $\alpha^\prime$ on $[a,b]$. Then the Riemann integral $\int_{a}^{b} f(x) \alpha^\prime(x)dx$ exists and we have
\[\int_{a}^{b} f(x) d\alpha(x) = \int_{a}^{b} f(x) \alpha^\prime(x)dx\]
%Riemann-Stieljes: 7.8, 7.9, 7.12 with examples
\end{theorem}
\begin{proof}
\end{proof}

\begin{theorem}[7.9]
Given $a < c < b$. Define $\alpha$ on $[a,b]$ as follows: The values $\alpha(a)$, $\alpha(c)$, $\alpha(b)$ are arbitrary;
\[\alpha(x) = \alpha(a) \textrm{ if } a \leq x < c\]
and
\[\alpha(x) = \alpha(b) \textrm{ if } c \leq x < b\]
Let $f$ be defined on $[a,b]$ in such a way that at least one of the functions $f$ or $\alpha$ is continuous from the left at $c$ and at least one is continuous from the right at $c$. Then $f \in R(\alpha)$ on $[a,b]$ and we have
\[\int_{a}^{b} f d\alpha = f(c)[\alpha(c+) - \alpha(c-)]\]
%Riemann-Stieljes: 7.8, 7.9, 7.12 with examples
\end{theorem}
\begin{proof}
\end{proof}

\begin{theorem}[7.12]
Every finite sum can be written as a Riemann-Stieltjes integral. In fact, given a sum $\sum_{k=1}^{n} a_{k}$, define $f$ on $[0,n]$ as follows:
\[f(x) = a_{k}  \textrm{  if  }  k-1 < x \leq k\]
for $k=1,2,...,n$, and $f(0) = 0$. Then
\[\sum_{k=1}^{n}a_{k} = \sum_{k=1}^{n}f(k) = \int_{0}^{n}f(x)d[x]\]
where $[x]$ is the greatest integer $\leq x$.
%Riemann-Stieljes: 7.8, 7.9, 7.12 with examples
\end{theorem}
\begin{proof}
\end{proof}

The following three proofs are with respect to problem 9
\begin{theorem}[7.30]
%Theorem 7.30, 7.32 and 7.34 with examples
\end{theorem}
\begin{proof}
\end{proof}

\begin{theorem}[7.32]
%Theorem 7.30, 7.32 and 7.34 with examples
\end{theorem}
\begin{proof}
\end{proof}

\begin{theorem}[7.34]
%Theorem 7.30, 7.32 and 7.34 with examples
\end{theorem}
\begin{proof}
\end{proof}

\begin{theorem}
Thms 9.2 with many examples (non u.c. does not imply convergence etc...)
\end{theorem}

\begin{proof}
\end{proof}

\begin{theorem}
Convergence Test: Thms 8.22, 8.23,8.25,8.26 with Examples
\end{theorem}

\begin{proof}
\end{proof}

\begin{theorem}
Thm. 9.5 and Thm 9.6 with examples.
\end{theorem}

\begin{proof}
\end{proof}

\end{document}
