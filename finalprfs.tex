\documentclass[aps,pra,notitlepage,amsmath,amssymb,letterpaper,12pt]{revtex4-1}
\usepackage{amsthm,enumerate,amsmath}
\usepackage{graphicx}
%  Above uses the Americal Physical Society template for Physical Review A
%  as a reasonable and fully-featured default template

%  Below define helpful commands to set up problem environments easily
\newtheorem{theorem}{Theorem}
\newenvironment{problem}[2][Problem]{\begin{trivlist}
\item[\hskip \labelsep {\bfseries #1}\hskip \labelsep {\bfseries #2.}]}{\end{trivlist}}
\newenvironment{solution}{\begin{proof}[Solution]}{\end{proof}}

% --------------------------------------------------------------
%                   Document Begins Here
% --------------------------------------------------------------

\begin{document}

\title{Final Proofs for MATH450 Real Analysis}
\author{Quinn Gates, Evan Walker}
%\affiliation{Math 450, Schmid College of Science and Technology, Chapman University}
\date{\today}

\maketitle


\begin{theorem}[3.24 Bolzano-Weierstrass]
If a bounded set $S \subseteq \mathbb{R}^n$ contains infinitely many points, then there is at least one point in $\mathbb{R}^n$ which is an accumulation point of $S$.
%Application: Thm. 3.25 Cantor Intersection Theorem
\end{theorem}

\begin{proof}
(Case $n=1$,$\quad \mathbb{R}^{1}$)
Let $S \subset \mathbb{R}^{1}$ and $S$ bounded, $\Rightarrow$ since $S$ is bounded, $S$ lies completely within an interval $[a,b]$. Since $S$ contains infinitely many points,
$$[a,\frac{a+b}{2}] \quad \textrm{ and }  \quad [\frac{a+b}{2},b]$$
may either both contain infinitely many points or only one contains infinitely many points. Now define $[a_{1},b_{1}]$ to be one of the previously stated intervals with infinitely many points. Then
$$[a_{1},\frac{a_{1}+b_{1}}{2}] \quad \textrm{ and }  \quad [\frac{a_{1}+b_{1}}{2},b_{1}]$$
again may either both contain infinitely many points or only one contains infinitely many points. Repeating this process $k$ times, we arrive at an interval $[a_{k},b_{k}]$
such that $$sup(a_{k}) = inf(b_{k}) = x$$ where $x \in S$. We aim to show that $x$ is an accumulation point of $S$. Consider $k$ to be large enough such that $b_{k} - a_{k} < \frac{r}{2}$, $\Rightarrow [a_{k},b_{k}] \subseteq B(x,r)$. Since, by construction, $B(x,r)$ contains infinitely many points, $(B(x,r) - \{x\}) \cap S \ne \emptyset$ and hence $x$ is an accumulation point of $S$.
\newline{}
\newline{}
(Case $n=2$,$\quad \mathbb{R}^{2}$) Let $S \subset \mathbb{R}^{2}$ and $S$ bounded, $\Rightarrow$ since $S$ is bounded, $S$ lies completely within a 2-dimensional interval $J$ defined by the cartesian product of two 1-dimensional intervals.
$$J_{1} = [a,b] \times [a,b]$$
similar to the case for $n=1$ above, we continually divide the interval into halfs, choosing the interval with infinitely many points. Let $I_{1}^{(1)} = [a,b]$, and $I_{2}^{(1)} = [a,b]$, Then let the next division of the interval be $I_{1}^{(2)} = [a,\frac{a+b}{2}] \textrm{ or } [\frac{a+b}{2},b] = [a_{1},b_{1}]$, and $I_{2}^{(2)} = [a,\frac{a+b}{2}] \textrm{ or } [\frac{a+b}{2},b] = [a_{2},b_{2}]$, Again choosing whichever interval has infinitely many points. If they both contain infinitely many points, our choice is arbitrary. Hence, define $J_{n} = I_{1}^{n} \times I_{2}^{n}$. For a large enough $n$, we have that $J_{n}$ has the property that
$$sup(a_{1}^{(n)}) = inf(b_{1}^{(n)}) = x_{1} \quad \textrm{ and } \quad sup(a_{2}^{(n)}) = inf(b_{2}^{(n)}) = x_{2}$$
We assert that $x=(x_{1}, x_{2})$ is an accumulation point of $S$. Note that $b_{k}^{(n)}-a_{k}^{(n)} = \frac{a}{2^{n-2}}$ for $k = 1,2$. take a ball in $\mathbb{R}^{2}$, $B(x,r)$ and let n be large enough such that $\frac{a}{2^{n-2}} < \frac{r}{2}$. Hence $J_{n} \subseteq B(x,r)$, $\Rightarrow$ $B(x,r)$ also contains infinitely many points and we have that,
$$ (B(x,r)- \{x\}) \cap S \ne \emptyset $$
This proves that $x$ is an accumulation point of $S$.



\end{proof}


\begin{theorem}[3.27] \nonumber
Let $G = \{A_{1},A_{2},...\}$ denote the countable collection of all n-balls having rational radii and centers at points with rational coordinates. Assume $x \in \mathbb{R}^n$ and let $S$ be an open set in $\mathbb{R}^n$ and $x \in S$. Then $x \in A_{k} \subseteq S$ for some $k$. That is, $x$ is contained by some n-ball in $G$ which is contained by $S$.
%Application: Thm 3.28 (Lindeloff Covering Theorem)
\end{theorem}
\begin{proof}
Theorem 2.27 states that if G is a countable collection of countable sets, then
\[\cup_{k=1}^{n} A_{k} \qquad A_{k} \in G\]
is also countable. Since $x \in S$, and $S$ is open, we know that there exists an n-ball such that
\[B(x,r) \subseteq S\]
We want to find a rational point $y\in S$ such that $x \in B(y,R) \in G$, or equally stated by $B(y,R) \subseteq B(x,r) \subseteq S$. Let
$x=(x_{1},x_{2},...,x_{n})$ and let $y_{k}$ be a rational number such that
\[\left| y_{k} - x_{k} \right| < \frac{r}{4n} \qquad \textrm{ for each } k=1,2,3,...,n\]
Then
\[\|y-x\| \leq \mid y_{1}-x_{1} \mid + ... + \mid y_{n}-x_{n} \mid < \frac{r}{4}\]
Let us define $R$ as a rational number such that $\frac{r}{4} < R < \frac{r}{2}$. Then
$x \in B(y,R)$ since we found the distance between $y$ and $x$ to be less than $r/4$, so surely $x$ must be contained in $B(y,R)$. By the same reasoning, we have that $B(y,R) \subseteq B(x,r) \subseteq S$. since both $y$ and $R$ are rational by construction, we have that $B(y,R) \in G$. Hence, we've proved the assertion.
\end{proof}


\begin{theorem}[3.29 Heine-Borel]
Let $F$ be an open covering of a closed and bounded set $A \subseteq \mathbb{R}^n$. Then a finite subcollection of $F$ also covers $A$.
%Application: you choose
\end{theorem}

\begin{proof}
\end{proof}


\begin{theorem}[3.31]
Let $S \subseteq \mathbb{R}^n$. Then the following statements are equivalent
\begin{enumerate}[\upshape a)]
  \item $S$ is compact. %note a implies b by heineborel
  \item $S$ is closed and bounded.
  \item Every infinite subset of $S$ has an accumulation point in $S$.
\end{enumerate}
%comparison with metric spaces (without proof)
\end{theorem}

\begin{proof}
\end{proof}


The next 3 theorems are with respect to problem 5.
\begin{theorem}[theorem 4.16]
Let $f:S \to T$  be a function from one metric space $(S,d_{S})$ to another $(T,d_{T})$, and assume $p \in S$. Then $f$ is continuous at $p$ iff for every sequence $\{x_{n}\}$ in $S$ convergent to $p$, the sequence $\{f(x_{n})\}$ in $T$ converges to $f(p)$. Symbolically,
\[\lim_{n \to \infty} f(x_{n}) = f(\lim_{n \to \infty} x_{n})\]
%Continuous Functions in Metric Spaces: Thm. 4.16, Thm. 4.23, Thm. 4.25
\end{theorem}
\begin{proof}
($\Longrightarrow$) Assume that $f$ is continuous at $p$. The definition of continuity at a point states that for every $\epsilon > 0$ there is a $\delta >0$ such that,
\[d_{T}(f(x),f(p)) < \delta \qquad \textrm{  whenever  } d_{S}(x,p) < \delta\]
And if every sequence $\{x_{n}\} \rightarrow p$ in $S$, then  $\forall \delta > 0 \quad \exists N$ s.t. $$d_{S}(x_{n},p) < \delta, \qquad \textrm{ whenever } n \geq N$$
\end{proof}

\begin{theorem}[theorem 4.23]
Let $f:S \to T$  be a function from one metric space $(S,d_{S})$ to another $(T,d_{T})$. Then $f$ is continuous on $S$ iff for every open set $Y$ in $T$, the inverse image $f^{-1}(Y)$ is open on $S$.
%Continuous Functions in Metric Spaces: Thm. 4.16, Thm. 4.23, Thm. 4.25
\end{theorem}
\begin{proof}
($\Longrightarrow$) Assume $Y$ is open and $f$ is continuous, Since $Y$ is open we have that $B_{T}(y,\epsilon) \subseteq Y$ for some $\epsilon > 0$. Let $p \in f^{-1}(Y)$ and $y=f(p)$, And Since $f$ is continuous on $S$ we have that $f$ is continuous at $p$ and hence $\exists \delta > 0$ such that  $f(B_{S}(p,\delta)) \subseteq B_{T}(y,\epsilon)$. Hence,
\[B_{S}(p,\delta) \subseteq f^{-1}[f(B_{S}(p,\delta))] \subseteq f^{-1}[B_{T}(y,\epsilon)] \subseteq f^{-1}(Y)\]
Then $p$ is an interior point of $f^{-1}(Y)$ and by Definition 3.6, if a set contains all its interior points then it is open, and since $p$ can be any point in $f^{-1}(Y)$ we have that $f^{-1}(Y)$ contains all its interior points and hence is open.
\newline{}
($\Longleftarrow$) Conversely, assume $f^{-1}(Y)$ is open in $S$ for every open subset set $Y$ in $T$. Take $p \in S$ and again let $y=f(p)$, then we want to show $f$ is continuous at $p$. $\forall \epsilon > 0$, $B_{T}(y,\epsilon)$ is open in $T$, hence $f^{-1}(B_{T}(y,\epsilon))$ is open in $S$. Since $f^{-1}(y) = p$ ,  $p \in f^{-1}(B_{T}(y,\epsilon))$, then $\exists \delta > 0$ such that,
\[B_{S}(p,\delta) \subseteq f^{-1}(B_{T}(y,\epsilon)) \Rightarrow f(B_{S}(p,\delta)) \subseteq B_{T}(y,\epsilon)\]
By definition 4.15, a function $f$ is continuous at $p$ iff $\forall \epsilon > 0, \quad \exists \delta > 0$ such that $ f(B_{S}(p,\delta)) \subseteq B_{T}(f(p),\epsilon)$, and hence we have that $f$ is continuous at $p$.
$$ $$
\end{proof}

\begin{theorem}[theorem 4.25]
Let $f:S \to T$  be a function from one metric space $(S,d_{S})$ to another $(T,d_{T})$. If $f$ is continuous on compact subset $X$ of $S$, then the image $f(X)$ is a compact subset of $T$; in particular, $f(X)$ is closed and bounded in $T$.
%Continuous Functions in Metric Spaces: Thm. 4.16, Thm. 4.23, Thm. 4.25
\end{theorem}
\begin{proof}
By Definition 3.30, a set $S$ is compact iff every open covering of $S$ contains a finite subcovering, that is, a finite subcovering that also covers $S$. And by Theorem 3.38, if $S$ is compact then it is also closed and bounded.
\newline{}
Hence our goal is to find a finite open subcovering of $f(X)$ and the assertion will be proved. Define $F$ to be an open covering of $f(X)$, so that $f(X) = \cup_{A \in F}A$. Since $f$ is continuous on $S$ and hence every subset of $S$, we can apply Theorem 4.23 to see that $\forall A \in F$, $f^{-1}(A)$ is also open in $(X,d_{S})$. Hence, $X \subseteq \cup_{A \in F}f^{-1}(A) $. Since $X$ is compact, we know by the definition of a compact set that there exists an open subcovering of $X$, hence $X \subseteq \cup_{k = 1}^{n}f^{-1}(A_{k}) $ for $A_{k} \in F$. And hence,
$$ f(X) \subseteq f(\cup_{k = 1}^{n}f^{-1}(A_{k})) = \cup_{k = 1}^{n}f(f^{-1}(A_{k})) = \cup_{k = 1}^{n}A_{k} $$
$\Rightarrow$ $f(X) \subseteq\cup_{k = 1}^{n}A_{k}$ and we have found our open subcovering, and hence $f(X)$ is closed and compact in $T$.
$$ $$
\end{proof}


\begin{theorem}[Uniform Continuity Thm. 4.47 (Heine)]
Let $f:S \to T$ be a function from one metric space $(S,d_{S})$ to another $(T,d_{T})$. Let $A$ be a compact subset of $S$ and assume that $f$ is continuous on $A$. Then $f$ is uniformly continuous on $A$.
\end{theorem}
\begin{proof}
Let $\epsilon > 0$ be given. Then each point $a \in A$ has associated with it a ball  $B_{S}(a,r)$, with $r$ depending on $a$ such that,
$$d_{T}(f(x),f(a)) < \frac{\epsilon}{2} \qquad \textrm{whenever } x\in B_{S}(a,r)\cap A$$
Consider the collection of balls $B_{S}(a,r/2)$. Since $A$ is compact, a finite number of them also cover $A$, say
$$ A \subseteq \cup_{k=1}^{n} B_{S}(a_{k},\frac{r_{k}}{2})  $$
in any ball with twice the radius, $B_{S}(a_{k},r_{k})$, we have
$$d_{T}(f(x),f(a_{k})) < \frac{\epsilon}{2} \qquad \textrm{whenever } x\in B_{S}(a_{k},r_{k}) \cap A$$
Let $\delta = min(r_{n}/2)$. We want to show that $\delta$ satisfies the condition of uniform continuity. Take $x \in A$ and $p \in A$ with $d_{S}(x,p) < \delta$. it is clear that $x \in B_{S}(a_{k},r_{k}/2)$, hence
$$ d_{T}(f(x),f(a_{k})) < \frac{\epsilon}{2} $$
By the Triangle inequality and our defined $\delta$, we have
$$d_{S}(p,a_{k}) \leq d_{S}(p,x) + d_{S}(x,a_{k}) < \delta + \frac{r_{k}}{2} \leq \frac{r_{k}}{2} + \frac{r_{k}}{2} = r_{k}$$
Hence, $d_{S}(p,a_{k}) \leq r_{k}$ and thus $p \in B_{S}(a_{k},r_{k}) \cap S$. Using the Triangle inequality and the fact that $d_{T}(f(p),f(a_{k})) < \epsilon /2$, we have
$$d_{T}(f(x),f(p)) \leq d_{T}(f(x),f(a_{k})) + d_{T}(f(a_{k}), f(p)) < \frac{\epsilon}{2} + \frac{\epsilon}{2} = \epsilon$$
hence we have that $d_{T}(f(x),f(p)) < \epsilon$ and $d_{S}(x,p) < \delta$, thus fulfilling the condition for uniform continuity.
$$ $$
\end{proof}

The following 2 theorems are with respect to problem 7
\begin{theorem}[Thm 5.10 (Rolle)]
Let $f$ be a functions, having a derivative (finite or infinite) at each point
of an open interval $(a,b)$ and continuous at the endpoints $a$ and $b$. if $f(a)=f(b)$, then there is at least one interior point c at which
\[f^\prime(c) = 0\]
%Application: Thm 5.11
\end{theorem}
\begin{proof}
Assume $f^\prime(x) \neq 0 \quad \forall x \in (a,b)$, we want to arrive at a contradiction. Since $f$ is continuous on a compact interval $(a,b)$, $\exists c_{1},c_{2} \in (a,b)$ such that $$f(c_{1}) = min(f(x)) \quad \textrm{ and } \quad f(c_{2}) = max(f(x))$$
by Theorem 5.9, $c_{1}, c_{2}$ cannot be interior points since this implies $f^\prime(c_{1}) = 0$ and $f^\prime(c_{2}) = 0$. This implies that $f(c_{1})$ and $f(c_{2})$ are endpoints. Since $f(a) = f(b) \Rightarrow f(c_{1}) = f(c_{2})$ and hence $f$ is constant on $[a,b]$. However this contradicts our assumption that $f^\prime(x) \neq 0 \quad \forall x \in (a,b)$. Therefore $f^\prime(c) = 0$ for some $c \in (a,b)$.
\end{proof}

\begin{theorem}[Thm 5.12 (Generalized Mean Value)]
Let $f$ and $g$ be two functions, each having a derivative (finite or infinite) at each point of an open interval $(a,b)$ and each continuous at the endpoints $a$ and $b$. Assume also that there is no interior point $x$ at which both $f^\prime(x)$ and $g^\prime(x)$ are infinite. Then for some interior point $c$ we have
\[f^\prime(c)[g(b)-g(a)] = g^\prime(c)[f(b)-f(a)]\]
%Application: Thm 5.11
\end{theorem}
\begin{proof}
Let $h(x) = f(x)[g(b) - g(a)] - g(x)[f(b) - f(a)]$. Since $h(x)$ is a linear combination of $f(x)$ and $g(x)$, $h(x)$ inherits many of the assumptions about $f(x)$ and $g(x)$, which is exactly what we want in order to use Rolle's theorem. Note that,
\begin{align*}
$h(a)$ &= $ f(a)[g(b) - g(a)] - g(a)[f(b) - f(a)]$
\\ &= $ f(a)g(b) - g(a)f(b)$
\\ &= $ - f(b)g(a) + g(b)f(a)$
\\ &= $ f(b)[g(b) - g(a)] - g(b)[f(b) - f(a)]$
\\ &= $ h(b)$
\end{align*}
So $h(a) = h(b)$, $ \Rightarrow $ by Rolle's theorem we have that $\exists $ $c$ such that $ h^\prime(c) = 0 $. And Hence,
\[h^\prime(c) = f^\prime(c)[g(b) - g(a)] - g^\prime(c)[f(b) - f(a)] = 0\]
$ \Rightarrow $
\[f^\prime(c)[g(b)-g(a)] = g^\prime(c)[f(b)-f(a)]\]
This proves the assertion.
\end{proof}

%\section{The following three proofs are with respect to problem 8}
The following three proofs are with respect to problem 8
\begin{theorem}[7.8]
Assume $f \in R(\alpha)$ on $[a,b]$ and assume that $\alpha$ has a continuous derivative $\alpha^\prime$ on $[a,b]$. Then the Riemann integral $\int_{a}^{b} f(x) \alpha^\prime(x)dx$ exists and we have
\[\int_{a}^{b} f(x) d\alpha(x) = \int_{a}^{b} f(x) \alpha^\prime(x)dx\]
%Riemann-Stieljes: 7.8, 7.9, 7.12 with examples
\end{theorem}
\begin{proof}
Let $g(x) = f(x)\alpha^\prime(x)$ and consider a Riemanm sum
$$S(P,g) = \sum_{k=1}^{n}g(t_{k})\Delta x_{k} = \sum_{k=1}^{n}f(t_{k})\alpha^\prime(t_{k})\Delta x_{k}$$
Using the same $P$ and $t_{k}$ we can form the following Riemann-Stieljes integral
$$S(P,f,\alpha) =  \sum_{k=1}^{n}f(t_{k})\Delta \alpha_{k}$$
From the M.V.T. we can write,
$$\Delta \alpha_{k} = \alpha^\prime(v_{k}) \Delta x_{k} \qquad \textrm{whenever } v_{k} \in (x_{k-1},x_{k}) $$
$\Longrightarrow$
$$S(P,f,\alpha)-S(P,g) =  \sum_{k=1}^{n}f(t_{k})\alpha^\prime(v_{k}) \Delta x_{k} - \sum_{k=1}^{n}f(t_{k})\alpha^\prime(t_{k})\Delta x_{k} $$
$$= \sum_{k=1}^{n}f(t_{k})[\alpha^\prime(v_{k})-\alpha^\prime(t_{k})]\Delta x_{k}$$
Since $f$ is bounded, $\left|f(x)\right| \leq M \quad \forall x \in [a,b]$, where $M > 0$. Continuity of $\alpha^\prime$ on $[a,b]$ implies uniform continuity on $[a,b]$.
$\Rightarrow \textrm{ for } \epsilon > 0, \quad \exists \delta > 0 $ such that
$$ 0 \leq \left| x - y \right| < \delta \quad \textrm{implies} \quad \left| \alpha^\prime(x)- \alpha^\prime(y) \right| < \frac{\epsilon}{2M(b-a)}$$
If we take a partition $P_{\epsilon}^{\prime}$ with norm $\|P_{\epsilon}^{\prime}\| < \delta$, then for any finer partition $P$ we will have
$$\left| \alpha^\prime(v_{k})- \alpha^\prime(t_{k}) \right| < \frac{\epsilon}{2M(b-a)} \Rightarrow \left| S(P,f,\alpha)-S(P,g) \right| < \frac{\epsilon}{2}$$
Since $f \in R(\alpha)$ on $[a,b]$, $\exists P_{\epsilon}^{\prime \prime}$ such that
$$ \left| S(P,f,\alpha)-\int_{a}^{b}fd\alpha \right| < \frac{\epsilon}{2} $$
Hence, combining the last two inequalities we have that
$$ \left| S(P,g)-\int_{a}^{b}fd\alpha \right| < \epsilon $$
This proves the assertion.
\end{proof}

\begin{theorem}[7.9]
Given $a < c < b$. Define $\alpha$ on $[a,b]$ as follows: The values $\alpha(a)$, $\alpha(c)$, $\alpha(b)$ are arbitrary;
\[\alpha(x) = \alpha(a) \textrm{ if } a \leq x < c\]
and
\[\alpha(x) = \alpha(b) \textrm{ if } c \leq x < b\]
Let $f$ be defined on $[a,b]$ in such a way that at least one of the functions $f$ or $\alpha$ is continuous from the left at $c$ and at least one is continuous from the right at $c$. Then $f \in R(\alpha)$ on $[a,b]$ and we have
\[\int_{a}^{b} f d\alpha = f(c)[\alpha(c+) - \alpha(c-)]\]
%Riemann-Stieljes: 7.8, 7.9, 7.12 with examples
\end{theorem}
\begin{proof}

\end{proof}

\begin{theorem}[7.12]
Every finite sum can be written as a Riemann-Stieltjes integral. In fact, given a sum $\sum_{k=1}^{n} a_{k}$, define $f$ on $[0,n]$ as follows:
\[f(x) = a_{k}  \textrm{  if  }  k-1 < x \leq k\]
for $k=1,2,...,n$, and $f(0) = 0$. Then
\[\sum_{k=1}^{n}a_{k} = \sum_{k=1}^{n}f(k) = \int_{0}^{n}f(x)d[x]\]
where $[x]$ is the greatest integer $\leq x$.
%Riemann-Stieljes: 7.8, 7.9, 7.12 with examples
\end{theorem}
\begin{proof}
\end{proof}

The following three proofs are with respect to problem 9
\begin{theorem}[7.30 first mean-value theorem for Riemann-Stieltjes integrals]
Assume that $\alpha$ is increasing and let $f \in R(\alpha)$ on $[a,b]$. Let $M$ and $m$ denote, respectively, the $sup$ and $inf$ of the set $\{f(x):x \in [a,b]\}$. Then $\exists c \in \mathbb{R}$ satisfying $m \leq c \leq M$ such that
\[\int_{a}^{b}f(x)d\alpha(x) = c\int_{a}^{b}d\alpha(x) = c[\alpha(a)-\alpha(b)]\]
In particular, if $f$ is continuous on $[a,b]$, then $c=f(x_{0})$ for some $x_{0} \in [a,b]$.
%Theorem 7.30, 7.32 and 7.34 with examples
\end{theorem}
\begin{proof}
\end{proof}

\begin{theorem}[7.32]
Let $\alpha$ be of bounded variation on $[a,b]$ and assume that
$f \in R(\alpha)$ on $[a,b]$. Define F by
\[F(x)=\int_{a}^{x} f d \alpha \textrm{ if } x \in [a,b]\]
\begin{enumerate}[\upshape a)]
  \item $F$ is of bounded variation $[a,b]$.
  \item Every point of continuity of $\alpha$ is also a point of continuity of $F$.
  \item if $\alpha$ is increasing on $[a,b]$, then $\exists F^\prime(x)$ at each point $x \in [a,b]$ where $\alpha^\prime(x)$ exists and where $f$ is continuous. For such $x$, we have $$F^\prime(x) = f(x)\alpha^\prime(x)$$
\end{enumerate}
%Theorem 7.30, 7.32 and 7.34 with examples
\end{theorem}
\begin{proof}
\end{proof}

\begin{theorem}[7.34 Second fundamental theorem of integral calculus]
Assume that $f \in R$ on $[a,b]$. Let $g$ be a function defined on $[a,b]$ such that $\exists g^\prime \in (a,b)$ and has the value
\[g^\prime(x) = f(x) \textrm{ for every } x \in (a,b)\]
At the endpoints assume that $g(a+)$ and $g(b-)$ exist and satisfy
\[g(a) - g(a+) = g(b) - g(b-)\]
Then we have
\[\int_{a}^{b} f(x)dx = \int_{a}^{b} g^\prime(x)dx = g(b) - g(a)\]
%Theorem 7.30, 7.32 and 7.34 with examples
\end{theorem}
\begin{proof}
\end{proof}

The next 4 theorems are with respect to problem 11
\begin{theorem}[8.22]
If $|x| < 1$, the series $1 + x + x^2 + ... $  converges and has the sum $\frac{1}{1-x}$. If $|x| \geq 1$, the series diverges.
%Convergence Test: Thms 8.22, 8.23,8.25,8.26 with Examples
\end{theorem}
\begin{proof}
\end{proof}

\begin{theorem}[8.23 integral test]
Let $f$ be a positive decreasing function defined on $[1, \infty)$ such that $\lim_{x \to +\infty}f(x) = 0$. For $n = 1,2,...$, define
\[s_{n} = \sum_{k=1}^{n}f(k), \qquad  t_{n} = \int_{1}^{n}f(x)dx,  \qquad  d_{n}=s_{n}-t_{n}\]
Then we have:
\begin{enumerate}[\upshape a)]
  \item $0 < f(n+1) \leq d_{n+1} \leq d_{n} \leq f(1)$, \qquad for  $n=1,2,...$
  \item $\lim_{n \to \infty}d_{n}$ exists
  \item $\sum_{n=1}^{\infty}f(n)$ converges iff the sequence ${t_{n}}$ converges
  \item $0 \leq d_{k}-\lim_{n \to \infty}d_{n} \leq f(k)$, \qquad for $k=1,2,...$
\end{enumerate}
%Convergence Test: Thms 8.22, 8.23,8.25,8.26 with Examples
\end{theorem}
\begin{proof}
\end{proof}

\begin{theorem}[8.25 ratio test]
Given a series $\sum a_{n}$ of nonzero complex terms, let
\[r = \lim_{n \to \infty} inf \left| \frac{a_{n+1}}{a_{n}}\right|, \qquad R = \lim_{n \to \infty} sup \left| \frac{a_{n+1}}{a_{n}}\right|\]
\begin{enumerate}[\upshape a)]
  \item The series $\sum a_{n}$ converges absolutely if $R < 1$.
  \item The series $\sum a_{n}$ diverges if $r>1$.
  \item The test is inconclusive if $r \leq 1 \leq R$.
\end{enumerate}
%Convergence Test: Thms 8.22, 8.23,8.25,8.26 with Examples
\end{theorem}
\begin{proof}
\end{proof}

\begin{theorem}[8.26 root test]
Given a series $\sum a_{n}$ of complex terms, let
\[p = \lim_{n \to \infty} sup \sqrt[n]{\mid a_{n} \mid}\]
\begin{enumerate}[\upshape a)]
  \item The series $\sum a_{n}$ converges absolutely if $p < 1$.
  \item The series $\sum a_{n}$ diverges if $p>1$.
  \item The test is inconclusive if $p=1$.
\end{enumerate}
%Convergence Test: Thms 8.22, 8.23,8.25,8.26 with Examples
\end{theorem}
\begin{proof}
\end{proof}

%The next 2 theorems are with respect to problem 12
%\begin{theorem}[9.5 Cauchy condition for uniform convergence of series]
%The infinite series $\sum f_{n}(x)$ converges uniformly on $S$ iff $\forall\epsilon > 0$ there is an $N$ such that $n>N$ implies,
%\[\left| \sum_{k=n+1}^{n+p} f_{k}(x) \right| < \epsilon, \qquad \textrm{ for each } p = 1,2,... \textrm{ and } \forall x \in S\]
%Thm. 9.5 and Thm 9.6 with examples.
%\end{theorem}
%\begin{proof}
%\end{proof}

%\begin{theorem}[9.6 Weierstrass M-test]
%Let $\{M_{n}\}$ be a sequence of nonnegative numbers such that
%\[0 \leq \left| f_{n}(x) \right| \leq \{M_{n}\}, \qquad \textrm{ for } n = 1,2,... \textrm{ and } \forall x \in S\]
%Then $\sum f_{n}(x)$ converges uniformly on $S$ if $\sum \{M_{n}\}$ converges.
%Thm. 9.5 and Thm 9.6 with examples.
%\end{theorem}
%\begin{proof}
%\end{proof}

\end{document}
